\documentclass[a4paper]{article}


\usepackage{amsthm,amssymb,amsmath}
\usepackage{mathtools}
\usepackage{thm-restate}

\usepackage{graphicx}
\usepackage{subcaption}


\usepackage{iclr2022_conference,times}
\usepackage{url}
\usepackage{hyperref}
\usepackage{cleveref}

\usepackage{amartya_ltx}

\usepackage{natbib}

\graphicspath{{./figs/}}
\title{Where do Adversarial examples lie?}
\author{Antiquus S.~Hippocampus, Natalia Cerebro \& Amelie P. Amygdale \thanks{ Use footnote for providing further information
about author (webpage, alternative address)---\emph{not} for acknowledging
funding agencies.  Funding acknowledgements go at the end of the paper.} \\
Department of Computer Science\\
Cranberry-Lemon University\\
Pittsburgh, PA 15213, USA \\
\texttt{\{hippo,brain,jen\}@cs.cranberry-lemon.edu} \\
\And
Ji Q. Ren \& Yevgeny LeNet \\
Department of Computational Neuroscience \\
University of the Witwatersrand \\
Joburg, South Africa \\
\texttt{\{robot,net\}@wits.ac.za} \\
\AND
Coauthor \\
Affiliation \\
Address \\
\texttt{email}
}

\newcommand{\fix}{\marginpar{FIX}}
\newcommand{\new}{\marginpar{NEW}}
% \iclrfinalcopy
\begin{document}



\maketitle

\begin{abstract}
    Despite achieving low errors in various classification tasks,
    over-parameterized neural networks have been shown to be vulnerable to
    gradient-based adversarial attacks~(\citet{szegedy2013intriguing}). In this paper, we first show that these
    attacks are created by manipulating the components of low variance in the
    input space and propose a simple method to detect them.
\end{abstract}

\section{Introduction}
\label{sec:intro}
There has been multiple attempts at understanding the cause of adversarial vulnerability in neural networks.~\citet{cisse17a}

\section{Identifying Adversarial Examples}
\label{sec:ident}
Consider a $k$-class classification problem defined on a data distribution
\(\cD\) over \(\reals^d\times\cY\). The goal of learning is to produce a set of
parameters \(\theta\) such that a neural network \(f_{\theta}\), parameterised
by \(\theta\), obtains a small expected error~\(\risk{\cD}{f}\).

\begin{restatable}[Expected risk]{defn}{emprisk}
    \label{defn:exp_loss}
    For a distribution $\cD$, and a neural network $f_\theta$, the expected risk
    is defined as 
  \[  \risk{\cD}{f}=\bP_{\br{\vec{x},y}\sim\cD}\bs{f\br{\vec{x}}\neq y},\]
  \end{restatable}

However, there are two main challenges to his. First, despite being fixed, the
distribution \(\cD\) is unknown to the learner and hence, the learner cannot
directly optimise the expected risk. Instead, the learner is presented with an
$N$-sized training dataset \(S_N \sim \cD^N\) sampled i.i.d. from \(\cD\). A neural
network \(f_\theta\) is then trained to minimise the empirical classification
error \(\empRisk{h}{N}\) on \(S_N\) with the hope that a small empirical
classification error~\(\empRisk{h}{N}\) would also translate to low expected
risk \(\risk{\cD}{f}\). 
\begin{defn}[Empirical risk]
    \label{defn:emp_loss}
    If the training set consists of $\{(\vec{x}_i, y_i)\}$ for $i \in \{1\cdots
    N\}$ then the empirical loss is defined as
    \[\empRisk{h}{N} = \frac{1}{N} \sum_{i=1}^N \bI\bs{h\br{\vec{x}_i} \neq y_i}\]
  \end{defn}

  Second, it is also hard to estimate the expected error~\(\risk{\cD}{f}\) as
  the learner does not have access to the distribution \(\cD\). Instead the
  learner has access to a test dataset, another set of i.i.d. samples from
  \(\cD\). To test the performance of a neural network, its error on this test set is considered as a proxy for~\(\risk{\cD}{f}\).
   
  For many modern over-parameterised neural network, experiments have usually
  borne out that a low empirical error on the train set also leads to a low test
  error. However, recent
  works~\citep{goodfellow2014explaining,szegedy2013intriguing} have shown that
  despite achieving low test error these neural networks are susceptible to
  \(\ell_p\) adversarial attacks and consequently, has a very high \(\ell_p\)
  adversarial error.

\begin{restatable}[$\ell_p$ Adversarial
  Error]{defn}{advrisk}\label{defn:adv_risk} For any distribution $\cD$
  defined over $\br{\vec{x},y}\in\reals^d\times\cY$, any classifier
  $h:\reals^d\rightarrow\cY$, and any $\gamma>0$, 
  %%
  the $\gamma$-\emph{adversarial} error is 
          \begin{equation} 
              \radv{\gamma}{h;\cD}=\bP_{\br{\vec{x},y} 
              \sim\cD}\bs{\exists \vec{z}\in\cB_{\gamma}
              \br{\vec{x}};h\br{\vec{z}}\neq y},
          \end{equation} 
  where $\cB_\gamma^p\br{\vec{x}}$ is the $\ell_p$ ball of radius $\gamma \ge
  0$ around $\vec{x}$ under the $\ell_p$ norm.
\end{restatable}

The observation that these state-of-the-art neural networks have a low test
error but a high adversarial error indicate that while they are well-behaved on
the manifold of the observed data~(low test error), their behavior is less
predictable if the data is moved slightly away from the manifold~(high
adversarial error). An algorithm that can detect whether there is a large mass away from the manifold can possibly detect adversarial examples. In this paper, we make this intuition more rigorous and present an algorithm to do the same.

More precisely, we define the {\em data subspace} as the space spanned by the
top eigenvectors of the data and the {\em adversarial subspace} as the space
orthogonal to it. Let the data matrix be \(S_{N\vert X}\in\reals^{N\times d}\)
and let \(\bc{c_1,\cdots,c_d}\) be the corresponding principal components in descending order of the variance of data explained by each component. For two positive integers \(1\le s_1\le s_2\le d\), the {\em data subspace} is defined as the space spanned by \(\{c_1,\cdots,c_{s_1}\}\) and the orthogonal space is the space spanned by \(c_{s_1+1},\cdots,c_{s_2}\). Then for every point \(\vec{x}\in\reals^d\), we assign a score 

\begin{equation}
  \rho\br{\vec{x}} = \sqrt{\dfrac{\sum_{i=s_1+1}^{s_2}\ip{\vec{c_i}}{\vec{x}}^2}{ \sum_{i=1}^{s_1} \ip{\vec{c_i}}{\vec{x}}^2}}
\end{equation}

Then for a given a threshold \(\rho_0\), if \(\rho\br{\vec{x}}\ge \rho_0\), we classify \(\vec{x}\) as an adversarial example otherwise we classify it as a natural example.


\bibliography{subspace_ref}
\bibliographystyle{plainnat}


\end{document}